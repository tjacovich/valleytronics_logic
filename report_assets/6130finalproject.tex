\documentclass{beamer}
\usetheme{PaloAlto}
\usepackage{float}
\usecolortheme{whale}
\title[Simulating Valleytronics Logic at the Gate Level] % (optional, only for long titles)
{Simulating Valleytronics Logic at Gate Level}
\subtitle{A C++ Library and Example for Building Fully Electric Valleytronic Processors}
\author[Author] % (optional, for multiple authors)
{T. Jacovich\inst{1}}
\institute[Universities Here and There] % (optional)
{
 % \inst{1}%
  %Department of Electrical and Computer Engineering\\
  %The George Washington University
  %\and
  \inst{1}%
  Department of Physics\\
  The George Washington University
}
\date[AM2018] % (optional)
{ECE 6120: Advanced Microarchitecture Final Presentation}
\subject{Computer Science}
\begin{document}
\frame{\titlepage}
\section[Introduction]{Introduction}
  \subsection[What is Valleytronics]{What is Valleytronics}
  \begin{frame}
    \frametitle{Introduction to Valleytronics}
    \framesubtitle{}
    
\begin{itemize}
  \item[$\bullet$]Valleytronics is an alternative to CMOS logic that takes advantage a unique property of certain classes of semi-metals.
  \vspace*{20pt}
  \item[$\bullet$]It is the subject of active research, with isolated circuits being produced, but large-scale production still a few years off.
\end{itemize}


  \end{frame}
    \subsection[Warning: Quantum Physics]{QM}
  \begin{frame}
    \frametitle{Valleytronics vs. CMOS}
    \small CMOS logic takes advantage of the Pauli Exclusion principle to create diodes and transistors. These allow easy voltage manipulation, but no additional constraints. Valleytronics offers a
    \begin{itemize}
  \item[$\bullet$] 
  
  \item[$\bullet$] Encryption/Decryption of Off-Chip Memory
  
  \item[$\bullet$] Secure Context Manager
\end{itemize}
\vspace*{20pt}
 
\vspace*{20pt}
  \end{frame}
  
\section[Reversible Logic]{Reversible}
\begin{frame}
\frametitle{Reversible Logic}
\framesubtitle{The Landauer Limit and Beyond}


\end{frame}
\subsection[Tamper-Evident Processing]{Tamper-Evident Processing}
\begin{frame}
\frametitle{Tamper-Evident Processing}
Verifiable computations need a way to identify and prevent operations from being tampered with. 


\end{frame}
\end{document}